\documentclass[12pt, a4paper]{article}

\usepackage[margin=2.5cm]{geometry}
\usepackage{graphicx}
\usepackage{booktabs}
\usepackage{array}
\usepackage{longtable}
\usepackage{multirow}
\usepackage{xcolor}
\usepackage{hyperref}
\usepackage{amsmath}
\usepackage{amssymb}
\usepackage{caption}
\usepackage{subcaption}
\usepackage{enumitem}
\usepackage{titlesec}
\usepackage{fancyhdr}
\usepackage{parskip}
\usepackage{mdframed}
\usepackage{tabularx}

\definecolor{insightbg}{RGB}{232, 244, 255}
\definecolor{insightborder}{RGB}{30, 120, 200}
\definecolor{titlegray}{RGB}{45, 55, 72}
\definecolor{accent}{RGB}{30, 120, 200}

\newmdenv[
  backgroundcolor=insightbg,
  linecolor=insightborder,
  linewidth=2pt,
  leftline=true, rightline=false, topline=false, bottomline=false,
  innerleftmargin=10pt, innerrightmargin=8pt,
  innertopmargin=6pt, innerbottommargin=6pt,
  skipabove=8pt, skipbelow=8pt
]{insightbox}

\titleformat{\section}{\large\bfseries\color{titlegray}}{}{0em}{\thesection\quad}
\titleformat{\subsection}{\normalsize\bfseries\color{titlegray}}{}{0em}{\thesubsection\quad}

\pagestyle{fancy}
\fancyhf{}
\rhead{\textcolor{gray}{\small Customer Segmentation Report}}
\lhead{\textcolor{gray}{\small Machine Learning 2 Project}}
\rfoot{\thepage}

\graphicspath{{figures/}}

\hypersetup{
  colorlinks=true,
  linkcolor=accent,
  urlcolor=accent,
  citecolor=accent,
  pdftitle={Customer Segmentation Report},
  pdfauthor={Omar Gamal ElKady}
}

\begin{document}

\begin{titlepage}
  \centering
  \vspace*{2cm}
  {\huge\bfseries\color{titlegray} Customer Segmentation\\[0.4em]
  for Credit Card Customers\par}
  \vspace{1cm}
  \rule{\textwidth}{1pt}
  \vspace{0.5cm}
  {\large Machine Learning 2 --- Project Report\par}
  \vspace{2cm}
  {\normalsize
    \textbf{Dataset:} CC General --- Credit Card Customer Behaviour\\[0.3em]
    \textbf{Algorithms:} K-Means, Gaussian Mixture Model, PCA, t-SNE
  }
  \vspace{1cm}
  {\large\bfseries Omar Gamal ElKady\par}
  \vspace{0.3cm}
  {\normalsize February 2026\par}
  \vfill
  \rule{\textwidth}{0.5pt}
\end{titlepage}

\tableofcontents
\newpage

\section{Executive Summary}

This report documents the end-to-end implementation of an unsupervised machine
learning pipeline that segments 8{,}950 credit card customers into seven
behaviourally distinct groups. The analysis follows five phases prescribed by the
project specification: data exploration and preprocessing, determination of the
optimal cluster count, customer segmentation, visualisation and analysis, and
actionable business recommendations.

\medskip
\noindent Key outcomes are summarised below.

\begin{center}
\begin{tabular}{lp{9cm}}
\toprule
\textbf{Phase} & \textbf{Key Outcome} \\
\midrule
Data Exploration    & 8{,}950 customers, 17 numeric features; 2 features with missingness;
                      strong right-skew; high collinearity in purchase-related features. \\
Preprocessing       & Median imputation; log1p transformation on 10 monetary/count features;
                      PCA retained 95\%+ variance in 6 components. \\
Optimal $k$         & Elbow method suggested $k=4$; silhouette score peaked at $k=7$;
                      $k=7$ selected. \\
Segmentation        & K-Means outperformed GMM (silhouette 0.4477 vs.\ 0.3259);
                      seven well-separated clusters identified. \\
Visualisation       & t-SNE, radar charts, and heatmaps confirm distinct cluster fingerprints. \\
\bottomrule
\end{tabular}
\end{center}

\section{Phase 1 --- Data Exploration \& Preprocessing}

\subsection{Dataset Overview}

The dataset contains \textbf{8{,}950 credit card customers} and \textbf{18 columns}
(one customer identifier + 17 numeric behavioural features). The features capture
six behavioural dimensions:

\begin{itemize}[noitemsep]
  \item \textbf{Balance behaviour:} \texttt{BALANCE}, \texttt{BALANCE\_FREQUENCY}
  \item \textbf{Purchase behaviour:} \texttt{PURCHASES}, \texttt{ONEOFF\_PURCHASES},
        \texttt{INSTALLMENTS\_PURCHASES}, \texttt{PURCHASES\_FREQUENCY},
        \texttt{ONEOFF\_PURCHASES\_FREQUENCY}, \texttt{PURCHASES\_INSTALLMENTS\_FREQUENCY},
        \texttt{PURCHASES\_TRX}
  \item \textbf{Cash advance behaviour:} \texttt{CASH\_ADVANCE}, \texttt{CASH\_ADVANCE\_FREQUENCY},
        \texttt{CASH\_ADVANCE\_TRX}
  \item \textbf{Credit \& payments:} \texttt{CREDIT\_LIMIT}, \texttt{PAYMENTS},
        \texttt{MINIMUM\_PAYMENTS}, \texttt{PRC\_FULL\_PAYMENT}
  \item \textbf{Tenure:} \texttt{TENURE}
\end{itemize}

\subsection{Missing Values}

\begin{figure}[h!]
  \centering
  \includegraphics[width=0.65\textwidth]{fig01_missing_values.png}
  \caption{Count of missing values per feature.}
  \label{fig:missing}
\end{figure}

Only two features contain missing values:

\begin{center}
\begin{tabular}{lrr}
\toprule
\textbf{Feature} & \textbf{Missing Count} & \textbf{Missing \%} \\
\midrule
\texttt{CREDIT\_LIMIT}     & 1   & 0.01\% \\
\texttt{MINIMUM\_PAYMENTS} & 313 & 3.50\% \\
\bottomrule
\end{tabular}
\end{center}

\begin{insightbox}
\textbf{Business Insight:} Both features are right-skewed monetary amounts. Median
imputation was applied rather than mean imputation to avoid inflating values and
introducing bias into the cluster centroids. The \texttt{CUST\_ID} column was
dropped before any modelling as it carries no behavioural information.
\end{insightbox}

\subsection{Feature Distributions \& Skewness}

\begin{figure}[h!]
  \centering
  \includegraphics[width=\textwidth]{fig02_feature_distributions.png}
  \caption{Raw distributions of all 17 numeric features before transformation.}
  \label{fig:distributions}
\end{figure}

\begin{figure}[h!]
  \centering
  \includegraphics[width=\textwidth]{fig15_skewness.png}
  \caption{Skewness of each feature (red bars: $|\text{skew}|>1$).}
  \label{fig:skewness}
\end{figure}

Skewness analysis reveals that \textbf{15 out of 17 features} have $|\text{skew}|>1$.
The most extreme cases are:

\begin{center}
\begin{tabular}{lr}
\toprule
\textbf{Feature} & \textbf{Skewness} \\
\midrule
\texttt{MINIMUM\_PAYMENTS}       & 13.85 \\
\texttt{ONEOFF\_PURCHASES}       & 10.05 \\
\texttt{PURCHASES}               &  8.14 \\
\texttt{INSTALLMENTS\_PURCHASES} &  7.30 \\
\texttt{PAYMENTS}                &  5.91 \\
\texttt{CASH\_ADVANCE\_TRX}      &  5.72 \\
\texttt{CASH\_ADVANCE}           &  5.17 \\
\bottomrule
\end{tabular}
\end{center}

\begin{insightbox}
\textbf{Business Insight:} Most customers carry moderate balances and purchase
amounts, while a minority of high-value customers drives extreme values. This
long right tail is characteristic of real-world financial data and motivates the
use of a log transformation to compress outlier influence without discarding those
customers entirely.
\end{insightbox}

\subsection{Correlation Analysis}

\begin{figure}[h!]
  \centering
  \includegraphics[width=\textwidth]{fig03_correlation_heatmap.png}
  \caption{Lower-triangular correlation heatmap of all numeric features.}
  \label{fig:correlation}
\end{figure}

Key correlation findings:

\begin{itemize}[noitemsep]
  \item \texttt{PURCHASES} and \texttt{ONEOFF\_PURCHASES}: $r \approx 0.92$ --- one-off
        purchases constitute the majority of total purchases for most customers.
  \item \texttt{PURCHASES\_FREQUENCY} and \texttt{PURCHASES\_INSTALLMENTS\_FREQUENCY}:
        $r \approx 0.86$ --- frequent purchasers tend to buy in instalments.
  \item \texttt{CASH\_ADVANCE} and \texttt{CASH\_ADVANCE\_TRX}: $r \approx 0.91$ ---
        higher cash-advance amounts correlate with higher transaction counts.
\end{itemize}

\begin{insightbox}
\textbf{Business Insight:} High collinearity between purchase-related features adds
redundant information to the feature space and can distort distance-based clustering.
PCA (applied in Section~\ref{sec:pca}) collapses these correlated dimensions into
fewer orthogonal components, improving cluster quality.
\end{insightbox}

\subsection{Outlier Detection}

\begin{figure}[h!]
  \centering
  \includegraphics[width=\textwidth]{fig04_outlier_boxplots.png}
  \caption{Box plots of monetary features showing extreme outliers.}
  \label{fig:outliers}
\end{figure}

\begin{insightbox}
\textbf{Business Insight:} Extreme outliers --- customers with \texttt{CASH\_ADVANCE}
exceeding \$40{,}000 or \texttt{CREDIT\_LIMIT} up to \$30{,}000 --- represent real VIP
and high-risk customer segments, not data errors. Removing them would eliminate entire
behavioural groups. Log transformation compresses their scale while retaining them in
the analysis.
\end{insightbox}

\subsection{Preprocessing --- Log$_{1}$p Transformation}

\begin{figure}[h!]
  \centering
  \includegraphics[width=0.85\textwidth]{fig05_log_transform.png}
  \caption{Effect of log$_{1}$p transformation on the \texttt{BALANCE} feature.}
  \label{fig:logtransform}
\end{figure}

Log$_{1}$p ($\log(1+x)$) transformation was applied to the 10 most skewed
monetary and count features:

\begin{center}
\texttt{BALANCE, PURCHASES, ONEOFF\_PURCHASES, INSTALLMENTS\_PURCHASES,}\\
\texttt{CASH\_ADVANCE, CREDIT\_LIMIT, PAYMENTS, MINIMUM\_PAYMENTS,}\\
\texttt{CASH\_ADVANCE\_TRX, PURCHASES\_TRX}
\end{center}

The $+1$ offset ensures safe handling of zero values (log(0) is undefined). No
StandardScaler was applied because PCA is sensitive to variance, and preserving
the relative scale differences between features carries meaningful behavioural
information.

\subsection{Dimensionality Reduction --- PCA}
\label{sec:pca}

\begin{figure}[h!]
  \centering
  \includegraphics[width=\textwidth]{fig06_pca_variance.png}
  \caption{Scree plot (left) and cumulative explained variance (right) for PCA.}
  \label{fig:pca}
\end{figure}

Principal Component Analysis was applied to the log-transformed feature matrix
($8{,}950 \times 17$). \textbf{6 principal components} were sufficient to capture
\textbf{95\% of the total variance}, reducing the feature dimensionality from 17
to 6 while eliminating correlated noise. This compressed representation was used
as input to all clustering algorithms.

\section{Phase 2 --- Determining the Optimal Number of Clusters}

To determine the optimal number of clusters, two complementary metrics were
evaluated over $k \in \{2, 3, \ldots, 10\}$:

\begin{itemize}[noitemsep]
  \item \textbf{Elbow Method (Inertia):} Measures within-cluster sum of squared
        distances. The ``elbow'' --- the point of diminishing returns --- indicates $k$.
  \item \textbf{Silhouette Score:} Measures how similar each point is to its own
        cluster compared to others ($-1$ to $+1$; higher is better).
\end{itemize}

\begin{figure}[h!]
  \centering
  \includegraphics[width=\textwidth]{fig07_elbow_silhouette.png}
  \caption{Elbow curve (inertia) and silhouette scores across $k=2$ to $k=10$.}
  \label{fig:elbow}
\end{figure}

\begin{center}
\begin{tabular}{crr}
\toprule
$k$ & \textbf{Inertia} & \textbf{Silhouette Score} \\
\midrule
2  & 290{,}787 & 0.3706 \\
3  & 214{,}367 & 0.3796 \\
4  & 161{,}004 & 0.4010 \\
5  & 131{,}575 & 0.4209 \\
6  & 112{,}908 & 0.4313 \\
\textbf{7}  & \textbf{101{,}037} & \textbf{0.4477} \\
8  &  93{,}773 & 0.4061 \\
9  &  87{,}372 & 0.3487 \\
10 &  82{,}687 & 0.3355 \\
\bottomrule
\end{tabular}
\end{center}

\textbf{Decision justification:}
\begin{itemize}[noitemsep]
  \item The second derivative of the inertia curve peaks at $k=4$, indicating the
        traditional elbow point.
  \item The silhouette score peaks at $k=\mathbf{7}$ (0.4477) and drops sharply at
        $k=8$.
  \item $k=7$ was chosen because it maximises the silhouette score among the candidate
        values ($\{4, 7\}$), indicating the most internally coherent and externally
        well-separated clusters.
\end{itemize}

\begin{insightbox}
\textbf{Business Insight:} While $k=4$ offers a simpler model, it merges customer
groups with meaningfully different behaviours (e.g., cash-advance revolvers vs.\
installment shoppers). $k=7$ provides sufficient granularity to design distinct
marketing strategies for each segment without over-fragmenting the customer base.
\end{insightbox}

\section{Phase 3 --- Customer Segmentation}

\subsection{Algorithm Selection: K-Means vs. GMM}

Both K-Means and Gaussian Mixture Model (GMM) were trained with $k=7$ on the
6-dimensional PCA-transformed data.

\begin{figure}[h!]
  \centering
  \includegraphics[width=0.55\textwidth]{fig08_model_comparison.png}
  \caption{Silhouette score comparison: K-Means vs. GMM at $k=7$.}
  \label{fig:modelcomp}
\end{figure}

\begin{center}
\begin{tabular}{lcc}
\toprule
\textbf{Model} & \textbf{Silhouette Score} & \textbf{Selected?} \\
\midrule
K-Means ($k=7$) & \textbf{0.4477} & \checkmark \\
GMM (full covariance, $k=7$) & 0.3259 & \\
\bottomrule
\end{tabular}
\end{center}

K-Means was selected as the final model. This is consistent with the roughly
spherical cluster shapes visible in the t-SNE projection (Figure~\ref{fig:tsne}),
for which K-Means is well suited. GMM's flexibility in modelling elliptical shapes
did not translate into better separability here because the PCA-compressed space
already produces compact, isotropic clusters.

\subsection{Cluster Profiles}

The final K-Means model with $k=7$ assigns every customer to one of seven segments.
Below is a detailed narrative profile of each segment.

\paragraph{Cluster 0 --- Big Spenders / VIP (18.2\%, 1{,}633 customers)}
The highest-purchasing segment with a mean of \$2{,}683 in purchases per period,
the highest credit limits, and a notable 26\% full-payment rate. These customers
shop frequently (purchase frequency 0.84) and split purchases across both one-off
and instalment channels. They carry moderate balances and make substantial payments.

\paragraph{Cluster 1 --- Cash-Only Dependents (23.1\%, 2{,}068 customers)}
The largest and most at-risk segment. Mean purchases are effectively \$0 --- these
customers never use the card for shopping. Instead they rely entirely on cash
advances (\$1{,}994 mean), carry high balances (\$2{,}151), and have a near-zero
full-payment rate (4\%). This is the bank's highest-risk concentration.

\paragraph{Cluster 2 --- Installment Shoppers (21.2\%, 1{,}895 customers)}
The most financially prudent segment. Low balances (\$365), zero cash-advance usage,
and the highest full-payment rate (30\%). Purchases are moderate (\$499) and
predominantly in instalments. This group presents the lowest credit risk.

\paragraph{Cluster 3 --- High-Risk Heavy Users (10.5\%, 938 customers)}
The most financially active and highest-risk group. Highest balances (\$2{,}932),
highest cash advances (\$2{,}200), second-highest purchases (\$2{,}068), and highest
credit limits (\$5{,}964). Near-zero full-payment rate (7\%). This segment generates
maximum revenue but also maximum default risk.

\paragraph{Cluster 4 --- One-off \& Cash Advance Revolvers (8.9\%, 798 customers)}
Characterised by one-off purchase behaviour (frequency 0.28) combined with heavy
cash advance usage (\$2{,}023). High balances (\$2{,}354) and low full-payment rates
(5\%) indicate revolving debt. A hybrid risk profile.

\paragraph{Cluster 5 --- One-off Shoppers (12.7\%, 1{,}138 customers)}
Moderate activity with a focus on one-off purchases (\$836 mean, all one-off). No cash
advance usage. Moderate balances (\$747) and a 14\% full-payment rate. A transitional
segment with potential to migrate toward VIP behaviour.

\paragraph{Cluster 6 --- Installment \& Cash Advance Revolvers (5.4\%, 480 customers)}
The smallest and most complex segment. Combines instalment purchase habits (\$523
mean, predominantly instalment) with significant cash advance usage (\$1{,}994).
High balances (\$2{,}580) and near-zero full-payment rate (4\%) --- a financially
stretched group.

\section{Phase 4 --- Visualisation \& Analysis}

\subsection{t-SNE 2D Cluster Projection}

\begin{figure}[h!]
  \centering
  \includegraphics[width=\textwidth]{fig09_tsne.png}
  \caption{t-SNE 2D projection of the 8{,}950 customers coloured by cluster assignment.}
  \label{fig:tsne}
\end{figure}

t-SNE (t-Distributed Stochastic Neighbour Embedding) projects the 6-dimensional
PCA space into 2D while preserving local neighbourhood structure. The seven clusters
appear as clearly separated colour blobs with minimal overlap, which is consistent
with a silhouette score of 0.4477.

\begin{insightbox}
\textbf{Business Insight:} The spatial separation confirms that the 7 groups are
genuinely distinct in their financial behaviour --- not arbitrary partitions. Any
overlap between adjacent clusters (e.g.\ C3 and C0) reflects customers who share
partial behaviours such as high purchases in both groups, but differ in cash-advance
usage.
\end{insightbox}

\subsection{Cluster Size Distribution}

\begin{figure}[h!]
  \centering
  \includegraphics[width=0.85\textwidth]{fig10_cluster_sizes.png}
  \caption{Number of customers per segment with percentage labels.}
  \label{fig:sizes}
\end{figure}

\begin{insightbox}
\textbf{Business Insight:} Cash-Only Dependents (C1) is the largest segment at 23.1\%
--- representing the bank's greatest concentration of default risk in a single group.
Installment \& Cash Advance Revolvers (C6) at only 5.4\% is the smallest but most
financially complex segment, requiring specialised monitoring despite its small size.
\end{insightbox}

\subsection{Feature Distributions by Cluster}

\begin{figure}[h!]
  \centering
  \includegraphics[width=\textwidth]{fig11_feature_by_cluster.png}
  \caption{Overlaid histograms of 8 key features coloured by cluster.}
  \label{fig:histbycluster}
\end{figure}

\begin{insightbox}
\textbf{Business Insight:} \texttt{PURCHASES} and \texttt{CASH\_ADVANCE} provide the
clearest cluster separation. C1 spikes at exactly zero for purchases, while C0 spreads
far to the right. \texttt{TENURE} shows near-complete overlap across all clusters,
confirming that account age carries no meaningful discriminatory power for segmentation.
\end{insightbox}

\subsection{Cluster Heatmap --- Mean Feature Values}

\begin{figure}[h!]
  \centering
  \includegraphics[width=\textwidth]{fig12_cluster_heatmap.png}
  \caption{Heatmap of normalised mean feature values per cluster. Raw means shown as
           annotations.}
  \label{fig:heatmap}
\end{figure}

\begin{insightbox}
\textbf{Business Insight:} C3 (High-Risk Heavy Users) is the darkest row across
\texttt{BALANCE}, \texttt{CASH\_ADVANCE}, and \texttt{PAYMENTS} simultaneously --- the
most financially active and highest-risk segment. C2 (Installment Shoppers) is the
lightest row overall --- lowest activity with zero cash-advance usage. The
\texttt{PRC\_FULL\_PAYMENT} column is meaningfully above zero only for C2, confirming
it is the only segment with a tendency to pay balances in full.
\end{insightbox}

\subsection{Radar Charts --- Cluster Fingerprints}

\begin{figure}[h!]
  \centering
  \includegraphics[width=\textwidth]{fig13_radar_charts.png}
  \caption{Normalised radar charts showing the behavioural profile of each cluster.}
  \label{fig:radar}
\end{figure}

\begin{insightbox}
\textbf{Business Insight:} C3 has the largest radar polygon, spiking simultaneously
across \texttt{PURCHASES}, \texttt{CASH\_ADVANCE}, and \texttt{BALANCE} --- the only
cluster combining all three risk dimensions. C1 has a highly asymmetric shape --- large
only on \texttt{CASH\_ADVANCE} and \texttt{BALANCE}, completely flat on
\texttt{PURCHASES}. C2 has the smallest polygon overall, reflecting the lowest financial
activity across all features.
\end{insightbox}

\subsection{Box Plots --- Key Financial Features by Segment}

\begin{figure}[h!]
  \centering
  \includegraphics[width=\textwidth]{fig14_boxplots.png}
  \caption{Box plots (outliers hidden) of 8 key features disaggregated by cluster.}
  \label{fig:boxplots}
\end{figure}

\begin{insightbox}
\textbf{Business Insight:}
\begin{itemize}[noitemsep]
  \item \textbf{BALANCE} --- C3 and C6 have the highest and widest boxes, confirming heavy
        debt carriers. C2 sits near zero with a narrow box.
  \item \textbf{PURCHASES} --- C0 has the highest median; C1 has a flat box at zero ---
        every customer in this segment truly never purchases.
  \item \textbf{CASH\_ADVANCE} --- C1, C3, C4, C6 all have elevated boxes; C0 and C2 sit
        at zero.
  \item \textbf{CREDIT\_LIMIT} --- C0 and C3 have the highest limits, reflecting the bank
        rewarding high spenders with more credit.
  \item \textbf{PRC\_FULL\_PAYMENT} --- Almost all clusters are near zero; C2 is the only
        segment with a visible upward spread.
  \item \textbf{PURCHASES\_FREQUENCY} --- C0 is narrow and near 1.0 (buys every month);
        C1 is flat at 0.0 (never buys). The sharpest contrast of all features.
  \item \textbf{TENURE} --- All clusters have nearly identical boxes --- account age
        provides no discriminatory value.
\end{itemize}
\end{insightbox}

\section{Phase 5 --- Business Insights \& Strategic Recommendations}

\subsection{Segment-Level Recommendations}

\begin{longtable}{p{3.5cm} p{3cm} p{8cm}}
\toprule
\textbf{Segment} & \textbf{Business Priority} & \textbf{Recommended Actions} \\
\midrule
\endfirsthead
\toprule
\textbf{Segment} & \textbf{Business Priority} & \textbf{Recommended Actions} \\
\midrule
\endhead
\midrule
\multicolumn{3}{r}{\small\itshape Continued on next page\ldots}
\endfoot
\bottomrule
\endlastfoot

\textbf{C0 --- Big Spenders (VIP)}
\newline (18.2\%)
& Revenue maximisation
& Cross-sell premium travel and lifestyle cards. Offer VIP concierge services, airport lounge access, and elevated reward multipliers. Maintain credit-limit headroom to enable continued high-spend behaviour. Monitor for competitor card switching. \\[4pt]

\textbf{C1 --- Cash-Only Dependents}
\newline (23.1\%)
& Risk mitigation
& Flag for proactive credit counselling. Offer personal loan products at lower interest rates than cash-advance fees as a debt-restructuring incentive. Set conservative credit-limit increase thresholds. Trigger alerts when cash-advance frequency rises. \\[4pt]

\textbf{C2 --- Installment Shoppers}
\newline (21.2\%)
& Cross-sell \& activation
& These low-risk customers are under-utilised. Offer instalment-specific promotional rates (e.g., 0\% instalment for 12 months). Upsell to premium card tiers. Target with partner merchant offers to increase purchase frequency. \\[4pt]

\textbf{C3 --- High-Risk Heavy Users}
\newline (10.5\%)
& Dual-track: revenue + risk
& Highest revenue potential but also highest default risk. Reward high spend with premium benefits while simultaneously stress-testing credit exposure. Implement early-warning delinquency models calibrated specifically to this segment. Limit unsolicited credit-limit increases. \\[4pt]

\textbf{C4 --- One-off \& Cash Advance Revolvers}
\newline (8.9\%)
& Debt management
& Offer structured repayment plans to reduce revolving cash-advance balances. Market balance-transfer promotions. Educate on the cost differential between purchase interest and cash-advance interest rates. \\[4pt]

\textbf{C5 --- One-off Shoppers}
\newline (12.7\%)
& Engagement \& migration
& Activate with bonus rewards on one-off spend categories (e.g., dining, travel). Introduce instalment features to convert single large purchases into recurring instalment transactions, deepening engagement. Migrate aspirational customers toward VIP tier. \\[4pt]

\textbf{C6 --- Installment \& Cash Advance Revolvers}
\newline (5.4\%)
& Specialist risk monitoring
& Despite small size, this segment's combined instalment and cash-advance usage creates complex credit exposure. Assign dedicated relationship managers. Offer debt consolidation products. Apply stricter credit review cycles. \\[4pt]
\end{longtable}

\subsection{Portfolio-Level Insights}

\begin{enumerate}
  \item \textbf{Risk concentration:} 44.9\% of customers (C1 + C3 + C6) carry high
        balances combined with heavy cash-advance reliance and near-zero full-payment
        rates. This is the primary default-risk concentration in the portfolio.

  \item \textbf{Revenue concentration:} 28.7\% of customers (C0 + C3) generate the
        highest payment and purchase volumes. Retaining these customers and expanding
        their credit products is the primary revenue-protection priority.

  \item \textbf{Under-leveraged segment:} C2 Installment Shoppers (21.2\%) are the
        safest customers yet have the lowest credit limits and lowest utilisation. They
        represent the greatest opportunity for risk-free credit expansion.

  \item \textbf{Tenure is irrelevant:} Account age does not differentiate customer
        behaviour. Loyalty programmes should be behaviour-based (spend patterns, payment
        discipline) rather than tenure-based.

  \item \textbf{Two-dimensional risk model:} The clearest risk stratification comes from
        combining \texttt{CASH\_ADVANCE} (liquidity stress indicator) with
        \texttt{PRC\_FULL\_PAYMENT} (repayment discipline indicator). Customers high on
        the former and low on the latter form the highest-risk cohort (C1, C3, C6).
\end{enumerate}

\section{Summary of Findings}

\begin{center}
\begin{tabular}{lp{10cm}}
\toprule
\textbf{Metric} & \textbf{Value / Outcome} \\
\midrule
Dataset size               & 8{,}950 customers, 17 behavioural features \\
Missing values             & 2 features; median-imputed \\
Preprocessing              & Log$_{1}$p on 10 features; PCA to 6 components (95\% variance) \\
Optimal $k$                & 7 (silhouette 0.4477, elbow at 4) \\
Best model                 & K-Means (silhouette 0.4477 vs GMM 0.3259) \\
Cluster count              & 7 distinct segments \\
Largest segment            & C1 Cash-Only Dependents (23.1\%) \\
Highest-risk segment       & C3 High-Risk Heavy Users + C1 Cash-Only Dependents \\
Highest-revenue segment    & C0 Big Spenders (VIP) + C3 High-Risk Heavy Users \\
Safest segment             & C2 Installment Shoppers (30\% full-payment rate) \\
Key discriminators         & PURCHASES, CASH\_ADVANCE, PRC\_FULL\_PAYMENT, PURCHASES\_FREQUENCY \\
Non-discriminating feature & TENURE (identical distribution across all clusters) \\
\bottomrule
\end{tabular}
\end{center}

\end{document}
